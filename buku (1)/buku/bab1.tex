\chapter{PENDAHULUAN}
\pagenumbering{arabic}
\vspace{1ex}

\section*{}
Penelitian ini dilatarbelakangi oleh berbagai kondisi yang menjadi acuan. Selain itu juga terdapat beberapa permasalahan yang akan dijawab sebagai luaran dari penelitian.
\vspace{1ex}

\section{Latar belakang}
\vspace{1ex}

Telekardiologi berasal dari kata tele yang berarti jarak jauh dan kardiologi yang merupakan suatu cabang kedokteran yang berhubungan dengan studi dan perawatan kelainan-kelainan di sistem kardiovaskular, yaitu jantung, pembuluh darah, dan pembuluh nadi\cite{cit:1}. Penyakit jantung merupakan penyakit yang menjadi penyebab utama kematian secara global dalam 15 tahun terakhir. Dari 56,9 juta kematian di seluruh dunia pada tahun 2016, lebih dari setengah (54\%) disebabkan oleh 10 penyebab teratas. Penyakit jantung iskemik dan stroke adalah pembunuh terbesar di dunia, yang apabila digabungkan jumlahnya menyebabkan  15,2 juta kematian pada tahun 2016 \cite{cit:2}. Penyakit kardiovaskular juga menjadi penyebab kematian nomor satu di Indonesia. Data dari Institute for Health Metrics and Evaluation, lembaga statistik kesehatan asal Amerika Serikat menyebutkan jumlah kematian akibat penyakit ini mencapai 36,3 persen dari total kematian di Indonesia pada 2016. Selanjutnya, kanker dan diabetes menjadi penyakit yang juga menimbulkan banyak kematian. Indonesia sendiri mendapat peringkat ke-3 di ASEAN setelah Laos dan Filiphine dalam hal jumlah kematian yang disebabkan penyakit kardiovaskular pada tahun 2016 \cite{cit:3}. Salah satu cara untuk mengurangi resiko kematian yang disebabkan penyakit kardiovaskular adalah melakukan pemeriksaan jantung sejak dini. Pemeriksaan kesehatan jantung dapat menggunakan Electrocardiogram (ECG).\

\hspace{2mm} Electrocardiogram adalah tes yang menggunakan alat elektrokardiograf yang dapat menunjukkan seberapa cepat jantung seseorang berdetak, dimana denyut jantung tersebut berupa impulse listrik yang diukur oleh sensor. ECG digunakan untuk menampilkan aktivitas pada jantung, sehingga tenaga medis dapat mendiagnosis apakah detak jantung pada pasien tersebut normal atau tidak. Pemeriksaan rekaman ECG biasanya hanya dapat dilakukan
dilakukan di rumah sakit dengan fasilitas lengkap. Hal tersebut membuat pasien jantung menjadi malas untuk memeriksakan kesehatan jantungnya ke Rumah Sakit. Terlebih lagi disaat kondisi pandemi COVID 19. Rumah Sakit merupakan salah satu tempat yang beresiko menyebabkan penyebaran virus COVID 19. Maka dari itu dibutuhkan sebuah sistem telekardiologi menggunakan aplikasi android agar pasien dan dokter dapat melakukan pemeriksaan secara online tanpa harus pergi ke Rumah Sakit. 

\vspace{1ex} 

\section{Perumusan Masalah}
\vspace{1ex}

Pemeriksaan rekaman ECG masih dilakukan secara offline di Rumah Sakit. Kondisi pandemi COVID 19 menyebabkan Rumah Sakit menjadi tempat yang beresiko penyebaran virus. Oleh karena itu, dibutuhkan sistem telekardiologi dengan menggunakan aplikasi android yang dapat digunakan pasien untuk mengambil sinyal ECG dan mengirim data sinyal ECG kepada dokter spesialis sehingga pasien dapat berkonsultasi dirumah pribadi.

\vspace{1ex}

\section{Tujuan}
\vspace{1ex}

Adapun tujuan dari penelitian Tugas Akhir ini adalah membuat membuat aplikasi android untuk mengambil data ECG melalui bluetooth dan mengirimkan data ECG tersebut kepada dokter spesialis jantung.
\vspace{1ex}

\section{Batasan masalah}
\vspace{1ex}
Batasan masalah yang timbul dari permasalahan Tugas Akhir ini adalah:
\vspace{1ex}
\begin{enumerate}[nolistsep]
	\item Tujuan utama penelitian ini adalah membuat aplikasi yang dapat merekam data ECG, mengirim data ECG, menampilkan grafik ECG, dan \textit{chatting}. 
	\vspace{1ex}
	
	\item Kegiatan uji adalah menampilkan gambar sinyal pada aplikasi android dan melakukan chatting antara pasien dengan dokter.
	\vspace{1ex}
	
	\item Data ECG diambil dari jantung penulis dan alat simulator.
		 
\end{enumerate}
\vspace{1ex}

\section{Sistematika Penulisan}
\vspace{1ex}
Laporan penelitian Tugas akhir ini tersusun dalam sistematika dan terstruktur sehingga mudah dipahami dan dipelajari oleh pembaca maupun seseorang yang ingin melanjutkan penelitian ini. Alur sistematika penulisan laporan penelitian ini yaitu:
\vspace{1ex}

\begin{enumerate}[nolistsep]
	\item BAB I Pendahuluan

	Bab ini berisi uraian tentang latar belakang permasalahan, penegasan dan alasan pemilihan judul, sistematika laporan, tujuan dan metodologi penelitian.
	\vspace{1ex}

	\item BAB II Dasar Teori

	Pada bab ini berisi tentang uraian secara sistematis teori-teori yang berhubungan dengan permasalahan yang dibahas pada penelitian ini. Teori-teori ini digunakan sebagai dasar dalam penelitian, yaitu informasi singkat mengenai sinyal ECG, \textit{Mobile Programming}, aritmia, dan elektrokardiogram.
	\vspace{1ex}

	\item BAB III Perancangan Sistem dan Impementasi

	Bab ini berisi tentang penjelasan-penjelasan terkait eksperimen yang akan dilakukan dan langkah-langkah data diolah hingga menghasilkan visualisasi. Guna mendukung eksperimen pada penelitian ini, digunakanlah blok diagram atau \textit{work flow} agar sistem yang akan dibuat dapat terlihat dan mudah dibaca untuk implementasi pada pelaksaan tugas akhir.
	\vspace{1ex}

	\item BAB IV Pengujian dan Analisa

	Bab ini menjelaskan tentang pengujian eksperimen yang dilakukan terhadap data dan analisanya. Beberapa teknik visualisasi akan ditunjukan hasilnya pada bab ini dan dilakukan analisa terhadap hasil visualisasi dan informasi yang didapat dari hasil mengamati visualisasi yang tersaji.
	\vspace{1ex}

	\item BAB V Penutup

	Bab ini merupakan penutup yang berisi kesimpulan yang diambil dari penelitian dan pengujian yang telah dilakukan. Saran dan kritik yang membangun untuk pengembangkan lebih lanjut juga dituliskan pada bab ini.
\end{enumerate}
\vspace{1ex}