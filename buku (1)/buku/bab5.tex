\chapter{PENUTUP}
\vspace{1ex}

\section{Kesimpulan}
\vspace{1ex}
Pada penelitian ini, dilakukan pengambilan data ECG dengan menggunakan sensor AD8232 dan kemudian dikirimkan menuju aplikasi android melalui modul bluetooth HC-05. Pengambilan data tersebut dilakukan selama 15 menit. Setelah semua data diterima oleh aplikasi, data kemudian akan diupload menuju database. Data yang telah diupload di database dapat diakses oleh Dokter Spesialis agar dapat diberi diagnosa melalui aplikasi android. Aplikasi android juga memiliki fitur chat sehingga pasien dapat berkonsultasi dengan dokter.

Dari hasil pengujian yang sudah dilakukan pada bab sebelumnya, dapat ditarik beberapa kesimpulan sebagai berikut:
\begin{enumerate}[nolistsep]
	
	\item Aplikasi dapat merekam data maksimal dengan sampling rate
	sebesar 280 sampel/detik,

	
	\item Aplikasi dapat berjalan optimal saat merekam data dengan
	sampling rate dibawah 200 sampel/detik,

	
	\item Hasil pengujian untuk menghitung data yang hilang menunjukkan bahwa data yang hilang setiap sampling adalah kurang
	dari 1\%
	
	\item Aplikasi dapat merekam data selama 15 menit dengan menggunakan baudrate 57600 pada HC-05 dan delay 1 ms pada arduino,

	
	\item Durasi upload dengan sampling rate 275 sampai 280 memiliki
	rata-rata selama 3 menit 52 detik,
	
	\item Durasi upload data menuju database pada hasil pengujian telah yang dilakukan paling lama adalah 5 menit 5 detik dan
	yang paling cepat adalah 47 detik.
	

	

\end{enumerate}
\vspace{1ex}

\section{Saran}
\vspace{1ex}

Untuk pengembangan penelitian selanjutnya terdapat beberapa saran sebagai berikut :
\vspace{1ex}
\begin{enumerate}[nolistsep]
	\item Mengirim data ECG menuju aplikasi selain menggunakan bluetooth,
	\item Menambah fitur aplikasi untuk dapat mendeteksi adanya aritmia dari data ECG hasil rekaman.
\end{enumerate}

