\begin{center}
\Large\textbf{ABSTRAK}
\end{center}
\vspace{1ex}

\begin{adjustwidth}{-0.2cm}{}
\begin{tabular}{lcp{0.6\linewidth}}
Nama Mahasiswa &:& Robby Aldriyanto Raffly \\
Judul Tugas Akhir &:& Aplikasi Telekardiologi untuk Mobile Android Dilengkapi Sensor ECG \\
Pembimbing &:& 1. Arief Kurniawan, ST., MT. \\
& & 2. Dr. I Ketut Eddy Purnama, ST., MT.  \\
\end{tabular}
\end{adjustwidth}
\vspace{1ex}

\setlength{\parindent}{0cm} Jantung merupakan organ yang sangat penting bagi tubuh. Kematian karena penyakit jantung merupakan salah satu kematian terbanyak di dunia. Penyakit jantung dapat diketahui sejak dini dengan cara memonitoring sinyal detak jantung. Pada Tugas Akhir ini kami mengajukan judul penelitian tentang implementasi sistem telekardiologi. Implementasi sistem telekardiologi yang kami ajukan adalah membuat aplikasi android yang dapat digunakan untuk mengambil data sinyal ECG dari arduino melalui modul bluetooth HC-05. Sinyal ECG tersebut didapatkan arduino dengan menggunakan sensor AD8232. Aplikasi juga memiliki fitur untuk mengirimkan sinyal ECG kepada dokter spesialis jantung sehingga dokter dapat mendiagnosa sinyal ECG pasien tanpa harus mengecek pasien dengan bertemu secara langsung. Aplikasi juga memiliki fitur chat dengan dokter agar pasien dapat berkonsultasi jarak jauh. Hasil yang diharapkan dari penelitian ini adalah aplikasi dapat mengambil data melalui komunikasi bluetooth, mengirimkan data sinyal ECG kepada dokter, dan chatting.

\vspace{2ex}

Kata Kunci : Telekardiologi, ECG, Pengolahan Sinyal, Aplikasi Mobile
\newpage